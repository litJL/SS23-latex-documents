\section*{Aufgabe 2}
\subsection*{a)}
\#Tafeln Schokolade = 12 \\
\#Pakete = 4 \\
Betrachte Anzahl aller ungeordneten Zahlenpartitionen \\
Gesucht: $ P(12,4) $\\
Anwenden der folgenden Formel (Satz 1.10.8):
\begin{align*}
  P(n,m) = P(n-m,m) + P(n-1,m-1)
\end{align*}
Beachte: $ P(n,n) = 1 $, $ P(n,1) = 1 $ und $ P(n,m) = 0\ \text{mit } n<m $.
Also genau so wie wirs schon gemacht haben (Antwort ist lang, schreibe es nicht nochmal hin). \\
$P(5,3) = |\{1,1,3\} \cup \{1,2,2\}|$
\pagebreak
\subsection*{b)}
!Nichtmehr ungeordnet!\\
z.B. $ 47 \rightarrow 0047 $ \\
Beachte Ziffern von 0-9 $\rightarrow$ 1-10 \\
Suchen somit Ziffern mit Quersumme 17 \\
Verwende Formel aus Satz 1.10.13 \\
\begin{align*}
  \binom{16}{3} = 560
\end{align*}
Es ist aber noch nicht sichergestellt, dass nur Ziffern von 1-10 verwendet werden.
Es kann nur eine Ziffer größer als 10 werden. \\
Für Ziffer 14 die Anzahl der Möglichkeiten:
\begin{align*}
  4\cdot \binom{2}{2} = 4
\end{align*}
Für Ziffer 13:
\begin{align*}
  4\cdot \binom{3}{2} = 12
\end{align*}
Für Ziffer 12:
\begin{align*}
  4\cdot \binom{4}{2} = 24
\end{align*}
Für Ziffer 11:
\begin{align*}
  4\cdot \binom{5}{2} = 40
\end{align*}
Der Binomialkoeffizient verwendet ebenfalls die Formel von oben (Restliche Ziffern die nicht z.B. 14 sind aufteilen, bei 14 gibt es nur einsen.)
Also 80 Illegale Möglichkeiten, somit ist die Letztendliche Anzahl $ 560 - 80 = 480 $.


\pagebreak
\section*{Aufgabe 3}
$ N_i \rightarrow i $-te aufgedeckte Karte steht\\
Für $ i=1,\ldots,n $:\\
\begin{align*}
&M_i=\begin{cases}
1&N_i=max_j = 1,\ldots,n\quad N_j\\
0&\text{sonst}
\end{cases}\\
&i \rightarrow M_i=1\\
&S=\sum_{i=1}^{n}M_i\\
&\text{Linearität des Erwartungswerts} \rightarrow E(S)=\sum_{i=1}^{n}E(M_i)\\
&E(M_i)=P(M_i=1)=\frac{1}{i}\\
\end{align*}
Die Wahrscheinlichkeit, dass das ausgewählte $ N_i $ die größte Zahl ist, ist $ \frac{1}{i} $, da es $ i $ Karten gibt und jede gleich wahrscheinlich ist.\\
\begin{align*}
&E(M_i)=\frac{1}{i}\\
&E(S)=\sum_{i=1}^{n}\frac{1}{i}\quad n \rightarrow \infty\\
\end{align*}
z.B. $ n=4 \rightarrow E(S)= 2.083 $

\pagebreak
\section*{Aufgabe 2}
\subsection*{a)}
Es ist zu zeigen, dass:
\begin{align*}
  \sum_{j=0}^{\infty}(1-r)^j\cdot r = 1
\end{align*}
Wir können umstellen:
\begin{align*}
r\cdot \sum_{j=0}^{\infty}(1-r)^j
\end{align*}
Da $0<r<1$ ist gilt $ 1-r < 1 $ weswegen wir folgende Formel für Reihen anwenden können:
\begin{align}
  \sum_{i=1}^{\infty}\frac{ c }{ 1-c } \text{ wenn }|c| < 1
\end{align}
Davor muss jedoch $ j=1 $ gelten:
\begin{align*}
  r\cdot ((1-r)^0 + \sum_{j=1}^{\infty}(1-r)^j) = r\cdot (1+ \sum_{j=1}^{\infty}(1-r)^j) 
\end{align*}
Mit (1) erhalten wir dann:
\begin{align*}
  r\cdot (1 + \frac{ 1-r }{ \cancel{1}-(\cancel{1}-r) } ) = r + \frac{ \cancel{r}\cdot(1-r) }{ \cancel{r} } = r+1-r = 1
\end{align*}
Also ist die Summe aller möglichen Wahrheitswerte von $ m_1 $ gleich 1. $ m_1 $ ist somit eine Wahrscheinlichkeitsfunktion.
\subsection*{b)}
Es ist zu zeigen, dass:
\begin{align*}
  \sum_{j=0}^{n}\binom{n}{j}p^j(1-p)^{n-j} = 1
\end{align*}
Wir definieren $ p=p $ und $ q=(1-p) $. Mithilfe des Hinweises erhalten wir:
\begin{align*}
  \sum_{j=0}^{n}\binom{n}{j}p^j(1-p)^{n-j} = (p+(1-p))^n = 1^n = 1
\end{align*}
Somit ist die Summer aller möglichen Wahrheitswerte von $ m_2 $ gleich 1. $ m_2 $ ist somit auch eine Wahrscheinlichkeitsfunktion.
\section*{Aufgabe 4}
\subsection*{a)}
\subsection*{b)}
Wir haben eine Menge der Mitarbeiter M und wollen diese in 3 Bautrupps (Partitionen) einteilen. Somit ist die Anzahl der Möglichkeiten diese Bautrupps zu bilden $ S(7,3) $ (wir verwenden die Ergebnisse aus Aufgabe 3b)

\begin{align*}
  &S(3,2) = S(2,1) + 2\cdot S(2,2) = 1 + 2\cdot1 = 3 \\
  &S(4,2) = 7 \\
  &S(5,2) = S(4,1) + 2\cdot S(4,2) = 1 + 2\cdot7 = 15 \\
  &S(6,3) = S(5,2) + 3\cdot S(5,3) = 15 + 3\cdot25 = 90 \\
  &S(6,2) = S(5,1) + 2\cdot S(5,2) = 1 + 2\cdot15 = 31 \\
  &S(7,3) = S(6,2) + 3\cdot S(6,3) = 31 + 3\cdot90 = 301 \\
\end{align*}
Also $ S(7,3) $ = 301
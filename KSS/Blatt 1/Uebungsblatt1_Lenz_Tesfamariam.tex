\documentclass[12pt,german,a4paper]{article}

\usepackage[ngerman]{babel}
\usepackage{geometry} % alternatively type-area
\usepackage{fancyhdr}

% Font and Encoding Packages
% -----------------------------------------------------------------------
\usepackage[T1]{fontenc}    % enables symbols as ä, ö, ü, ß
\usepackage[utf8]{inputenc} % uft8-encoded symbols can processed
\usepackage{amsfonts}       % fonts mainly used in a mathematical context
\usepackage[lighttt]{lmodern}
%\usepackage[scaled=0.85]{DejaVuSansMono}

% Symbols
% -----------------------------------------------------------------------
% Specialized symbols
\usepackage{amssymb}	 % collection of mathematical symbols
\usepackage{extarrows}	 % collection of various arrows


% Maths Packages
% -----------------------------------------------------------------------
% These packages provide environments and commands for mathematical purposes
\usepackage{mathtools} % huge collection for mathematical writing replaces/extends amsmath
\usepackage{amsthm}    % provides claim-environments like theorem, lemma
\usepackage{cases}     % provides environment 'cases' for case distinctions
\usepackage{nicefrac}  % provides cmd 'nicefrac' to write '1/2' in texts


% Table and Picture Packages
% -----------------------------------------------------------------------
\usepackage{array}      % provides cmd to program talbes, e.g. with of a column
\usepackage{enumitem}   % user control over the basic list environments itemize, enumerate, and description
\usepackage{graphicx}   % provides optional arguments for 'includegraphics' like scaling
\usepackage{caption}    % enables description/captions for images, tables, algorithms
\usepackage{float}      % provides the 'H' option for floating objects like figures
\usepackage{pdfpages}   % enables the inclusion of PDFs in a latex document


% Computer Science Packages
% -----------------------------------------------------------------------
\usepackage[small]{complexity} % tool set for complexity theory
\usepackage{listings}          % integrate program source code into the pdf
\usepackage{algorithm}         % provides basic environment for algorithm display
\usepackage{algpseudocode}     % provides commands to write pseudocode
\usepackage{fancyvrb}


% Citation, Reference, Colours
% -----------------------------------------------------------------------
\usepackage{hyperref} % provides clickable in-document-references
\usepackage{xcolor}   % everything related to colours

\newcommand{\submissiontitle}[1]{
	\pagestyle{fancy}
	\setlength{\headheight}{14.5pt}
	\fancyhead[R]{\thepage}
	\fancyhead[L]{#1}
	\fancyfoot{}
	\title{#1}
}

\usepackage{cancel}

%%%%%%%%%%%%%%%%%%%%%%%%%%%%%%%%%%%%%%%%%%%%%%%%%%%%%%%%%%%%%%%%%%%%%%%%%
% REDUCING the white part
%%%%%%%%%%%%%%%%%%%%%%%%%%%%%%%%%%%%%%%%%%%%%%%%%%%%%%%%%%%%%%%%%%%%%%%%%

\newgeometry{headheight=15.5pt, marginparwidth=2cm}
%\linenumbers

% allow page-breaks in mathematics mode but avoid them
\allowdisplaybreaks[1]


%%%%%%%%%%%%%%%%%%%%%%%%%%%%%%%%%%%%%%%%%%%%%%%%%%%%%%%%%%%%%%%%%%%%%%%%%
% Hyperlinks Setup
%%%%%%%%%%%%%%%%%%%%%%%%%%%%%%%%%%%%%%%%%%%%%%%%%%%%%%%%%%%%%%%%%%%%%%%%%

\hypersetup{
	colorlinks = true,		% instead of frames around the links, colour links
	linkcolor  = darkblue,  % colour internal links
	citecolor  = darkred,	% colour of citations
	urlcolor   = darkgreen	% colour of external links
}

%%%%%%%%%%%%%%%%%%%%%%%%%%%%%%%%%%%%%%%%%%%%%%%%%%%%%%%%%%%%%%%%%%%%%%%%%
% Colours
%%%%%%%%%%%%%%%%%%%%%%%%%%%%%%%%%%%%%%%%%%%%%%%%%%%%%%%%%%%%%%%%%%%%%%%%%

\definecolor{darkgreen}{rgb}{0,0.6,0}
\definecolor{darkred}{RGB}{185,40,25}
\definecolor{darkblue}{RGB}{0,95,140}
\definecolor{darkyellow}{RGB}{155,135,12}

\submissiontitle{Blatt 1 - Gruppe 3}
\author{Mike Lenz, Jonas Tesfamariam} 

\begin{document}
\maketitle	
%--Content--

\section*{Aufgabe 1}
\begin{align}
	\binom{n}{m}\binom{m}{k}&=\binom{n}{k}\binom{n-k}{m-k} \\
  \frac{ n! }{ m!\cdot(n-m)! }\cdot \frac{ m! }{ k!\cdot(m-k)! } &= \frac{ n! }{ k!\cdot(n-k)! }\cdot \frac{ (n-k)! }{ (m-k)!\cdot((n-k)-(m-k))! } \\
	\frac{ n! }{ \cancel{m!}\cdot(n-m)! }\cdot \frac{ \cancel{m!} }{ k!\cdot(m-k)! } &= \frac{ n! }{ k!\cdot\cancel{(n-k)!} }\cdot \frac{ \cancel{(n-k)!} }{ (m-k)!\cdot((n-k)-(m-k))! } \\ 
	\frac{ n! }{ k!\cdot (n-m)!\cdot(m-k)! } &= \frac{ n! }{ k!\cdot (n-m)!\cdot(m-k)! } 
\end{align}
Da auf beiden Seiten bei (4) die gleiche Formel steht ist $\binom{n}{m}\binom{m}{k}=\binom{n}{k}\binom{n-k}{m-k}$ wahr. 
\pagebreak
\section*{Aufgabe 2}
\subsection*{a)}
$\binom{32}{6}$ ist die Anzahl aller möglichen Blätter und $\binom{8}{6}$ ist die Anzahl aller möglichen Blättern mit einer Farbe (Persönlich denken wir, dass z.B. Herz und Karo eine Farbe sein sollten aber in der Aufgabe ist jedes Symbol als eigene Farbe erwähnt). \\
Die Wahrscheinlichkeit ein einzelnes bestimmtes Blatt zu ziehen ist 
$\frac{ 1 }{ \binom{32}{6}}$
Womit die Wahrscheinlichkeit ein Blatt zu ziehen, bei welchem alle Karten die selbe Farbe haben,
$$
\frac{ 4\cdot\binom{8}{6} }{ \binom{32}{6} } \approx 0,0001 = 0,01\%
$$
ist. Wir multiplizieren mal 4 da jede Farbe betrachtet werden soll.
\subsection*{b)}
Für jeden Wert gibt es einen Drilling den man jedes Blatt ziehen kann, also gibt es 8 mögliche Drillinge pro Blatt. Da es 4 verschiedene Drillings-Kombinationen pro Farbe gibt multiplizieren wir mit 4. Wir brauchen lediglich zwei Drillinge, also gibt es
$4\cdot\binom{8}{2}$ verschiedene Blätter mit zwei Drillingen.
Die Wahrscheinlichkeit solch ein Blatt zu ziehen ist
$$
\frac{ 4\cdot\binom{8}{2} }{ \binom{32}{6} } \approx 0,0001 = 0,01\%
$$
\pagebreak
\section*{Aufgabe 3}
\subsection*{a)}
Da $21=3\cdot7$ ist $T _{ 21 } \subset T _{ 7 } $, also kann 21 ausgelassen werden. Zudem, da 7,9 und 15 Teilerfremd sind, können wir vorgehen wie im Beispiel 1.3.4 vom Skript. \\
Laut Siebformel ist
$$|T_7\cup T_9\cup T _{ 15 } | = |T_{7}|+|T_{9}|+|T_{15}|-|T_{7}\cap T_{9}|-|T_{7}\cap T_{15}|-|T_{9}\cap T_{15}|+|T_{7}\cap T_{9}\cap T _{ 15 }|.$$
Für $T_i$ gilt $|T_i|=\lfloor 10000/i\rfloor$ und wie in Beispiel 1.3.4 gilt $T_i \cap T_j = T_{i\cdot j}$. Also
\begin{align*}
&|T_{7}|=1428 \\
&|T_{9}|=1111 \\
&|T_{15}|=666 \\
&|T_{7}\cap T_{9}| = |T_{\text{kgV(7,9)}}| = |T_{63}| = 158 \\
&|T_{7}\cap T_{15}| = |T_{\text{kgV(7,15)}}| = |T_{105}| = 95 \\
&|T_{9}\cap T_{15}| = |T_{\text{kgV(9,15)}}| = |T_{45}| = 222 \\
&|T_{7}\cap T_{9}\cap T_{15}| = |T_{\text{kgV(7,9,15)}}| = |T_{315}| = 31 \\
\end{align*}
Somit ist
$$
|T_7\cup T_9\cup T _{ 15 } | = 1428+1111+666-158-95-222+31=2761
$$
\subsection*{b)}
Wir definieren die Mengen E,F und L, wessen Elemente die Lehrer sind, die die Sprache unterrichten. Im Schnitt sind Lehrer enthalten, welche beide Sprachen unterrichten, z.B. in $E\cap F$ sind alle Lehrer die Englisch und Französisch unterrichten. Aus dem Text können wir folgern, dass
\begin{align*}
&|E|=21 \\
&|F|=14 \\
&|L|=18 \\
&|E\cap L|=8 \\
&|F\cap L|=4 \\
&|E\cap F\cap L|=3 \\
&|E\cup F\cup L|=39. \\
\end{align*}
Hiermit ist
\begin{align*}
|E\cup F\cup L| &= |E| + |F| + |L| - |E\cap L| - |F\cap L| - |E\cap F| + |E\cap F\cap L| \\
39 &= 21+14+18-8-4-|E\cap F|+3 \\
39 &= 44 -|E\cap F| \\
39+|E\cap F| &= 44 \\
|E\cap F| &= 5.
\end{align*}
Also gibt es 5 Lehrer, welche Englisch und Französisch, aber kein Latein, unterrichten.
\pagebreak
\section*{Aufgabe 4}
/
\pagebreak
\section*{Aufgabe 5}
\lstinputlisting[language=Python, postbreak=\mbox{\textcolor{red}{$\hookrightarrow$}\space}, breaklines=true]{./code/aufgabe5.py}
Wir haben unseren Code in Python geschrieben.
Beispielausgaben: \\
Für 500 $\rightarrow$ 146 \\
Für 10000 $\rightarrow $ 2922 \\
Für 123456 $\rightarrow $ 36075
\end{document}
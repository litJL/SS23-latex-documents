\section*{Aufgabe 1}
Wir verwenden Bayes Umkehrformel $ P(A | B) = \frac{P(B | A) \cdot P(A)}{P(B)} $.\\
Die Wahrscheinlichkeiten für die Erkrankungen ist die Anzahl der Teilnehmer, die diese Krankheit haben geteilt durch die Anzahl der Teilnehmer. z.B.:
\begin{align*}
  P(A) = \frac{333}{1000} = 33.3\%
\end{align*}
Die Wahrscheinlichkeiten für die Testergebnisse sind gegeben durch die Summe der Einträge in der jeweiligen Spalte geteilt durch die Anzahl der Teilnehmer. z.B.:
\begin{align*}
  P(00) = \frac{221 + 3 + 51 + 50}{1000} = 32.5\%
\end{align*}
Und zuletzt die bedingten Wahrscheinlichkeiten für die Testergebnisse gegeben die Erkrankung. Hier teilen wir nicht durch die Anzahl aller Teilnehmer, sondern durch die Anzahl der Teilnehmer mit der jeweiligen Erkrankung. z.B.:
\begin{align*}
  P(00 | A) = \frac{221}{333} \approx 66.3\%
\end{align*}
Dies können wir nun Anwenden:
\begin{align*}
  P(A | 00) = \frac{ \frac{ 221 }{ 333 } \cdot \frac{ 333 }{ 1000 } }{ \frac{ 325 }{ 1000 }  } =  0.68
\end{align*}
Wir gehen Analog für die anderen Wahrscheinlichkeiten vor:
\begin{align*}
  P(B | 00) = \frac{ \frac{ 3 }{ 11 } \cdot \frac{ 11 }{ 1000 } }{ \frac{ 325 }{ 1000 }  } \approx 0.00923\\
\end{align*}
\begin{align*}
  P(C | 00) = \frac{ \frac{ 51 }{ 222 } \cdot \frac{ 222 }{ 1000 } }{ \frac{ 325 }{ 1000 }  } \approx  0.156\\
\end{align*}
\begin{align*}
  P(D | 00) = \frac{ \frac{ 50 }{ 434 } \cdot \frac{ 434 }{ 1000 } }{ \frac{ 325 }{ 1000 }  } \approx  0.153\\
\end{align*}
Für 10:
\begin{align*}
  P(A | 10) = \frac{ \frac{ 31 }{ 333 } \cdot \frac{ 333 }{ 1000 } }{ \frac{ 387 }{ 1000 }  } \approx  0.08\\
\end{align*}
\begin{align*}
  P(B | 10) = \frac{ \frac{ 5 }{ 11 } \cdot \frac{ 11 }{ 1000 } }{ \frac{ 387 }{ 1000 }  } \approx  0.0129\\
\end{align*}
\begin{align*}
  P(C | 10) = \frac{ \frac{ 20 }{ 222 } \cdot \frac{ 222 }{ 1000 } }{ \frac{ 387 }{ 1000 }  } \approx  0.0517\\
\end{align*}
\begin{align*}
  P(D | 10) = \frac{ \frac{ 331 }{ 434 } \cdot \frac{ 434 }{ 1000 } }{ \frac{ 387 }{ 1000 }  } \approx  0.855\\
\end{align*}
Für 01:
\begin{align*}
  P(A | 01) = \frac{ \frac{ 60 }{ 333 } \cdot \frac{ 333 }{ 1000 } }{ \frac{ 183 }{ 1000 }  } \approx  0.328\\
\end{align*}
\begin{align*}
  P(B | 01) = \frac{ \frac{ 2 }{ 11 } \cdot \frac{ 11 }{ 1000 } }{ \frac{ 183 }{ 1000 }  } \approx  0.011\\
\end{align*}
\begin{align*}
  P(C | 01) = \frac{ \frac{ 111 }{ 222 } \cdot \frac{ 222 }{ 1000 } }{ \frac{ 183 }{ 1000 }  } \approx  0.607\\
\end{align*}
\begin{align*}
  P(D | 01) = \frac{ \frac{ 10 }{ 434 } \cdot \frac{ 434 }{ 1000 } }{ \frac{ 183 }{ 1000 }  } \approx  0.0546\\
\end{align*}
Für 11:
\begin{align*}
  P(A | 11) = \frac{ \frac{ 21 }{ 333 } \cdot \frac{ 333 }{ 1000 } }{ \frac{ 105 }{ 1000 }  } = 0.2\\
\end{align*}
\begin{align*}
  P(B | 11) = \frac{ \frac{ 1 }{ 11 } \cdot \frac{ 11 }{ 1000 } }{ \frac{ 105 }{ 1000 }  } \approx  0.009\\
\end{align*}
\begin{align*}
  P(C | 11) = \frac{ \frac{ 40 }{ 222 } \cdot \frac{ 222 }{ 1000 } }{ \frac{ 105 }{ 1000 }  } \approx  0.381\\
\end{align*}
\begin{align*}
  P(D | 11) = \frac{ \frac{ 43 }{ 434 } \cdot \frac{ 434 }{ 1000 } }{ \frac{ 105 }{ 1000 }  } \approx  0.409\\
\end{align*}
\pagebreak
\section*{Aufgabe 3}
\subsection*{a)}
Wir setzten die Informationen aus der Aufgabenstellung in die Tschebyscheffsche Ungleichung ein:
\begin{align*}
  P(|S_n - \mu|\le 0.5) \ge 0.98 = \frac{ \sigma^2 }{n\cdot\varepsilon^2 }
\end{align*}
Durch umstellen erhalten wir:
\begin{align*}
  n = \frac{ \sigma^2 }{0.98 \cdot 0.5^2 }
\end{align*}
Wobei $ \sigma^2 $ folgendermaßen definiert ist:
\begin{align*}
  \sigma^2 = E(X_i^2) - E(X_i)^2
\end{align*}
Wir müssen lediglich eine Zufallsvariable $ X_i $ betrachten, da alle Zufallsvariablen gleich sind.
Wir berechnen $ E(X_i^2) $ und $ E(X_i)^2 $:
\begin{align*}
  E(X_i^2) &= \sum_{j=1}^{m} j^2 \cdot \frac{ 1 }{ m } = \frac{ 1 }{ m } \cdot \sum_{j=1}^{m}j^2 = \frac{ 1 }{ \cancel{m} } \cdot \frac{ \cancel{m}\cdot (m+1)(2m+1) }{ 6 }  = \frac{ (m+1)(2m+1) }{ 6 }
\end{align*}
\begin{align*}
  E(X_i)^2 &= \left(\sum_{j=1}^{m} j \cdot \frac{ 1 }{ m }\right)^2 = \left(\frac{ 1 }{ m } \cdot \sum_{j=1}^{m}j\right)^2 = \left(\frac{ 1 }{ \cancel{m} } \cdot \frac{ \cancel{m}\cdot (m+1) }{ 2 }\right)^2 = \left(\frac{ (m+1) }{ 2 }\right)^2 = \frac{ (m+1)^2 }{ 4 }\\
\end{align*}
Also:
\begin{align*}
  \sigma^2 &= E(X_i^2) - E(X_i)^2 = \frac{ (m+1)(2m+1) }{ 6 } - \frac{ (m+1)^2 }{ 4 } = \frac{ 2\cdot (2m^2+3m+ 1) }{ 12 } - \frac{ 3m^2 + 6m + 3 }{ 12 }\\ &=\frac{ m^2-1}{ 12 }
\end{align*}
Nun können wir für $ \sigma^2 $ einsetzen:
\begin{align*}
  n &= \frac{ \sigma^2 }{0.98 \cdot 0.5^2 } = \frac{ \frac{ m^2-1 }{ 12 } }{0.98 \cdot 0.5^2 } = \frac{ m^2-1 }{ 12 \cdot 0.98 \cdot 0.5^2 } = \frac{ m^2-1 }{ 2.94 }\\
\end{align*}
Die Anzahl an benötigten Versuchen lässt sich also aus mit $ m $ berechnen.
\subsection*{b)}
Wir setzen wieder die Informationen aus der Aufgabenstellung in die Tschebyscheffsche Ungleichung ein:
\begin{align*}
  P(|S_n - 0.5|\le 0.01) \ge 0.96 = \frac{ \sigma^2 }{n\cdot 0.01^2 }
\end{align*}
Durch umstellen erhalten wir:
\begin{align*}
  n = \frac{ \sigma^2 }{0.96 \cdot 0.01^2 }
\end{align*}
$ \sigma^2 $ ist wie in a) definiert:
\begin{align*}
  \sigma^2 = E(X_i^2) - E(X_i)^2
\end{align*}
Wir berechnen $ E(X_i^2) $ und $ E(X_i)^2 $:
\begin{align*}
  E(X^2) = 0^2 \cdot \frac{ 1 }{ 2 } + 1^2 \cdot \frac{ 1 }{ 2 } = 0.5
\end{align*}
\begin{align*}
  E(X)^2 = 0.5^2 = 0.25
\end{align*}
Also:
\begin{align*}
  \sigma^2 &= E(X_i^2) - E(X_i)^2 = 0.5 - 0.25 = 0.25
\end{align*}
Nun können wir für $ \sigma^2 $ einsetzen:
\begin{align*}
  n &= \frac{ \sigma^2 }{0.96 \cdot 0.01^2 } = \frac{ 0.25 }{0.96 \cdot 0.01^2 } = 2604.17
\end{align*}
Also sind 2605 Versuche nötig.
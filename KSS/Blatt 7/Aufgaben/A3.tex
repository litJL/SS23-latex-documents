\pagebreak
\section*{Aufgabe 3}
\subsection*{a)}
Wir verechnen jeweils den Anteil der Studierenden mit der respektiven Abschlussquote und summieren alles letztendlich auf.
\begin{align*}
  0.2 \cdot 0.61 + 0.35 \cdot 0.48 + 0.45 \cdot 0.35 = 0.4475
\end{align*}
Es schaffen also 44,75\% der Studierenden den Abschluss.
\subsection*{b)}
Wir verwenden Bayes Umkehrformel. Davor definieren wir noch die Ereignisse:\\
$G$ = Studierender studiert ein Geisteswissenschaftliches Fach. \\
$A$ = Studierender schafft den Abschluss. \\
\begin{align*}
  P(G|A) = \frac{P(A|G)\cdot P(G)}{P(A)} = \frac{0.61\cdot 0.2}{0.4475} \approx 0.273
\end{align*}
\textbf{P(G|A)} ist die Wahrscheinlichkeit dafür, dass ein Studierender der den Abschluss schafft, ein Geisteswissenschaftliches Fach studiert hat. \\
\textbf{P(A|G)} ist die Wahrscheinlichkeit dafür, dass ein Studierender, der ein Geisteswissenschaftliches Fach studiert hat, den Abschluss schafft. Dies ist im Text bereits erwähnt mit einer Wahrscheinlichkeit von 61\%. \\
\textbf{P(G)} ist die Wahrscheinlichkeit dafür, dass ein Studierender ein Geisteswissenschaftliches Fach studiert. Ebenfalls im Text erwähnt mit 20\%.\\
\textbf{P(A)} ist die Wahrscheinlichkeit dafür, dass ein Studierender den Abschluss schafft. Diese haben wir in a) berechnet mit 44.75\%. \\
Ein Studierender studiert also mit einer Wahrscheinlichkeit von 27.3\% ein Geisteswissenschaftliches Fach, wenn er den Abschluss schafft.


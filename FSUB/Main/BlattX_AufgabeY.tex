\documentclass[a4paper]{scrartcl}

\usepackage[utf8]{inputenc}
\usepackage[ngerman]{babel}

\usepackage{amsmath,amssymb,tabularx}
\usepackage[space,extendedchars]{grffile}
\usepackage{verbatim}
\usepackage{url,mathrsfs,enumerate,dsfont}
\usepackage{algorithm,algorithmic}
\usepackage{listings}
\usepackage{geometry}
\usepackage{tikz}
\usepackage{etoolbox}
\usetikzlibrary{automata,arrows}
\usepackage{fancyhdr}
\usepackage{subfigure}
\usepackage{hyperref}
\usepackage{framed}
\usetikzlibrary{arrows,shapes, automata}
\setkomafont{disposition}{\normalfont\bfseries}
\setlength\parindent{0pt}

%%%%%%%%%%%%%%%%%%%%%%%%%
% BITTE HIER ANPASSEN.  %
\title{Blatt X, Aufgabe Y}
\subtitle{Gruppe K}
\author{Mike Lenz, Luca Winterkamp, Tim Zingler}
%%%%%%%%%%%%%%%%%%%%%%%%%


\date{}
\begin{document}
\maketitle

\begin{framed}
\textbf{WICHTIG!}
\begin{itemize}
\item Passen Sie die Nummer des Blatts und der Aufgabe, sowie die Informationen (Nummer und Namen) zu Ihrer Gruppe an.
 \item Bitte vergessen Sie nicht die pdf Datei zu generieren und \textbf{NUR} diese hochzuladen. .tex Dateien werden nicht akzeptiert!
 \item Sie können diese Vorlage beliebig anpassen, weitere Bibliotheken dazu nehmen, etc. Bitte achten Sie aber darauf, dass die Informationen zur Gruppe und zum Blatt erhalten bleiben.
\end{itemize}
\end{framed}
\pagebreak
test
\end{document}
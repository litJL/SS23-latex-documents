\pagebreak
\section*{Aufgabe \bn.1}
Der code sieht wie Python aus, wir nehmen aber an, dass hier Arrayinidizierung mit 1 beginnt.
\begin{Verbatim}[frame=single, label=Arrayanfang sortieren]
  def beginSort(A,k):
    n = len(A)

    quickSelect(A,1,n,k)
    
    quickSort(A,1,k-1)
\end{Verbatim}
\paragraph*{Korrektheit} Mithilfe von Quickselect suchen wir das k-te Element des Arrays, welches die obere Schranke unseres zu sortierenden Arrayanfangs markiert. Danach gilt für jedes $i,j$ mit $ 1 \le j < k < i \le n $:
\begin{align*}
  A[j] \le A[k] \le A[i]
\end{align*}
Nun muss lediglich der Array von 1 bis k-1 mit einem Algorithmus in $ O(n\log n) $ sortiert werden. Wir verwenden Quicksort mit zufälliger Pivot-wahl, da die erwartete Laufzeit des Algorithmus $ O(n\log n) $ ist. Nach der Ausführung von Quicksort gilt:
\begin{align*}
  A[1] \le \dots \le A[k-1] \le A[k]\quad \text{und}\quad A[k]\le A[i]\ \forall k < i \le n
\end{align*}
Dies ist die Bedingung aus der Aufgabenstellung.
\paragraph*{Laufzeit} QuickSelect läuft in O(n) und unsere Wahl von QuickSort läuft erwartet in $O(n\log n)$. Da jedoch nur von 1 bis $k-1$ sortiert wurde ist die Laufzeit in $ O(k\log k) $. Die Laufzeiten zusammen:
\begin{align*}
  O(n) + O(k\log k) = O(n + k\log k)
\end{align*}
Die Laufzeitbedingung ist somit erfüllt.
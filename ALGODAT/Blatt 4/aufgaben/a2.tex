\newpage
\section*{Aufgabe \bn.2}
\subsection*{a)}
\subsubsection*{k=2}
$ \lceil\log_{2}(2!)\rceil = 1 \rightarrow $ 1 Vergleich erlaubt.
\begin{Verbatim}[frame=single]
if A[0] >= A[1]:
  swap(A[0], A[1])
\end{Verbatim}
Klar zu sehen, dass nur ein Vergleich verwendet wird
\subsubsection*{k=3}
$ \lceil\log_{2}(3!)\rceil = 3 \rightarrow $ 3 Vergleiche erlaubt.
\begin{Verbatim}[frame=single]
if A[0] >= A[1]:
  swap(A[0], A[1])
if A[1] >= A[2]:
  swap(A[1], A[2])
if A[0] >= A[1]:
  swap(A[0], A[1])
\end{Verbatim}
Klar zu sehen, dass nur 3 Vergleiche verwendet werden.
\subsubsection*{k=4}
$ \lceil\log_{2}(4!)\rceil = 5 \rightarrow $ 5 Vergleiche erlaubt.
\begin{Verbatim}[frame=single]
if A[0] >= A[1]:
  swap(A[0], A[1])
if A[2] >= A[3]:
  swap(A[2], A[3])
if A[0] >= A[2]:
  swap(A[0], A[2])
if A[1] >= A[3]:
  swap(A[1], A[3])
if A[1] >= A[2]:
  swap(A[1], A[2])
\end{Verbatim}
Klar zu sehen, dass nur 5 Vergleiche verwendet werden.
\subsection*{b)}
/
\subsection*{c)}
\begin{Verbatim}[frame=single]
b = 0

for i in range(1,n):  //(n-1) Vergleiche
  if A[i] > A[b]:
    b = i
swap(A[b],A[n-1])

s = 0

for i in range(1,n-1) //(n-2) Vergleiche
  if A[i] < A[s]:
    s = i
swap(A[s],A[0])
\end{Verbatim}
Dieser Algorithmus macht (n-1) + (n-2) Vergleiche, jedoch nicht weniger als diese Anzahl.
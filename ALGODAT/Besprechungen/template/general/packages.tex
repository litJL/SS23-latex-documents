% !TEX root = ../main.tex
% !TEX spellcheck = en_GB

%%%%%%%%%%%%%%%%%%%%%%%%%%%%%%%%%%%%%%%%%%%%%%%%%%%%%%%%%%%%%%%%%%%%%%%%%
% PACKAGES
%%%%%%%%%%%%%%%%%%%%%%%%%%%%%%%%%%%%%%%%%%%%%%%%%%%%%%%%%%%%%%%%%%%%%%%%%

% Standard Packages
% -----------------------------------------------------------------------
\usepackage[ngerman]{babel}
\usepackage{geometry} % alternatively type-area
\usepackage{fancyhdr}


% Font and Encoding Packages
% -----------------------------------------------------------------------
\usepackage[T1]{fontenc}    % enables symbols as ä, ö, ü, ß
\usepackage[utf8]{inputenc} % uft8-encoded symbols can processed
\usepackage{amsfonts}       % fonts mainly used in a mathematical context
\usepackage[lighttt]{lmodern}
%\usepackage[scaled=0.85]{DejaVuSansMono}

% Symbols
% -----------------------------------------------------------------------
% Specialized symbols
\usepackage{amssymb}	 % collection of mathematical symbols
\usepackage{extarrows}	 % collection of various arrows


% Maths Packages
% -----------------------------------------------------------------------
% These packages provide environments and commands for mathematical purposes
\usepackage{mathtools} % huge collection for mathematical writing replaces/extends amsmath
\usepackage{amsthm}    % provides claim-environments like theorem, lemma
\usepackage{cases}     % provides environment 'cases' for case distinctions
\usepackage{nicefrac}  % provides cmd 'nicefrac' to write '1/2' in texts


% Table and Picture Packages
% -----------------------------------------------------------------------
\usepackage{array}      % provides cmd to program talbes, e.g. with of a column
\usepackage{enumitem}   % user control over the basic list environments itemize, enumerate, and description
\usepackage{graphicx}   % provides optional arguments for 'includegraphics' like scaling
\usepackage{caption}    % enables description/captions for images, tables, algorithms
\usepackage{float}      % provides the 'H' option for floating objects like figures
\usepackage{pdfpages}   % enables the inclusion of PDFs in a latex document


% Computer Science Packages
% -----------------------------------------------------------------------
\usepackage[small]{complexity} % tool set for complexity theory
\usepackage{listings}          % integrate program source code into the pdf
\usepackage{algorithm}         % provides basic environment for algorithm display
\usepackage{algpseudocode}     % provides commands to write pseudocode
\usepackage{fancyvrb}


% Citation, Reference, Colours
% -----------------------------------------------------------------------
\usepackage{hyperref} % provides clickable in-document-references
\usepackage{xcolor}   % everything related to colours


% Notes
% -----------------------------------------------------------------------
\usepackage[colorinlistoftodos]{todonotes}
\usepackage{lineno}     % to add line numbers for correction purposes


%%%%%%%%%%%%%%%%%%%%%%%%%%%%%%%%%%%%%%%%%%%%%%%%%%%%%%%%%%%%%%%%%%%%%%%%%
% CONFIGURATION for ALGORITHMS
%%%%%%%%%%%%%%%%%%%%%%%%%%%%%%%%%%%%%%%%%%%%%%%%%%%%%%%%%%%%%%%%%%%%%%%%%

\lstset{
    mathescape   = true,
	language     = c++,
	basicstyle   = \footnotesize\ttfamily,
	morekeywords = {Eingabe, Ausgabe},
	tabsize      = 4,
    numbers      = left,
    stepnumber   = 1,
    xleftmargin  = 4.5ex
}

%%%%%%%%%%%%%%%%%%%%%%%%%%%%%%%%%%%%%%%%%%%%%%%%%%%%%%%%%%%%%%%%%%%%%%%%%
% ToDo NOTES
%%%%%%%%%%%%%%%%%%%%%%%%%%%%%%%%%%%%%%%%%%%%%%%%%%%%%%%%%%%%%%%%%%%%%%%%%

\newcommand{\todoinline}[1]{\todo[inline, caption = {}]{#1}}


%%%%%%%%%%%%%%%%%%%%%%%%%%%%%%%%%%%%%%%%%%%%%%%%%%%%%%%%%%%%%%%%%%%%%%%%%
% ENVIRONMENT - THEOREM
%%%%%%%%%%%%%%%%%%%%%%%%%%%%%%%%%%%%%%%%%%%%%%%%%%%%%%%%%%%%%%%%%%%%%%%%%

% Theorem with numbers
\theoremstyle{definition}
\newtheorem{theorem}             {Theorem}
\newtheorem{lemma}      [theorem]{Lemma}
\newtheorem{corollary}  [theorem]{Korollar}
\newtheorem{claim}      [theorem]{Behauptung}

\theoremstyle{definition}
\newtheorem{definition} [theorem]{Definition}
\newtheorem{problem}	[theorem]{Problem}
\newtheorem{example}    [theorem]{Beispiel}
\newtheorem{remark}     [theorem]{Anmerkung}
\newtheorem{observation}[theorem]{Beobachtung}



%%%%%%%%%%%%%%%%%%%%%%%%%%%%%%%%%%%%%%%%%%%%%%%%%%%%%%%%%%%%%%%%%%%%%%%%%
% SIMPLE MACROS
%%%%%%%%%%%%%%%%%%%%%%%%%%%%%%%%%%%%%%%%%%%%%%%%%%%%%%%%%%%%%%%%%%%%%%%%%

% Mathematical Sets
\newcommand{\Rel}{\ensuremath{\mathbb{R}}}
\newcommand{\Rat}{\ensuremath{\mathbb{Q}}}
\newcommand{\Int}{\ensuremath{\mathbb{Z}}}
\newcommand{\Nat}{\ensuremath{\mathbb{N}}}
\newcommand{\Bool}{\ensuremath{\mathbb{B}}}

% Mathematics macros
\newcommand{\eqText}[1]{\ensuremath{\mathrel{\overset{\text{\sffamily\scriptsize #1}}{=}}}}

% Computer Science Marcos
\newcommand{\calO}[1]{\ensuremath{\mathcal{O}(#1)}}
\newcommand{\reducesMOto}[1]{\leqslant^{#1}_{\lang{mo}}}



%%%%%%%%%%%%%%%%%%%%%%%%%%%%%%%%%%%%%%%%%%%%%%%%%%%%%%%%%%%%%%%%%%%%%%%%%
% ENUMERATE ENVIRONMENTS
%%%%%%%%%%%%%%%%%%%%%%%%%%%%%%%%%%%%%%%%%%%%%%%%%%%%%%%%%%%%%%%%%%%%%%%%%

% The environments redefined here provide a alternative to the common enumeration (enumerate) without adjusting the label by hand each time 


% roman enumerate environment -- (i)
\newenvironment{enumrom}{
	\begin{enumerate}[labelindent=\parindent, leftmargin=*, label=(\roman*), align=left]
	}{
	\end{enumerate}
	}

% numbered enumerate environment -- (1)
\newenvironment{enumnum}{
	\begin{enumerate}[leftmargin=*, label=(\arabic*), align=left]
	}{
	\end{enumerate}
	}

% case enumerate environment -- 1. case:
\newenvironment{enumcase}{
	\begin{enumerate}[labelindent=.5em, leftmargin=*, label=Case~\arabic*:, align=left]
	}{
	\end{enumerate}
	}

% case enumerate environment -- 1. case:
\newenvironment{parts}{
	\begin{enumerate}[labelindent=.5em, leftmargin=*, label=\alph*), align=left]
	}{
	\end{enumerate}
	}


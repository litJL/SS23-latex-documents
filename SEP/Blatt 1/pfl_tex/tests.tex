\chapter{Systemtestfälle}

Hier sollen verschiedene Szenarien beschrieben werden, mithilfe deren Sie später Systemtests ausführen und die erwarteten Ergebnisse darstellen.

\newcounter{tf}\setcounter{tf}{10}

\begin{description}[leftmargin=5em, style=sameline]

\begin{lhp}{tf}{TF}{tests:anmelden}
	\item [Name:] Spieler anmelden.
	\item [Motivation:] Testet, ob die Anmeldung in das System korrekt funktioniert.
	\item [Sczenarien:] \hfill
		\begin{enumerate}
			\item \textit{Zugriffsdaten sind vorhanden und richtig} \\ $\implies$ Spieler wird auf die Hauptseite bewegt.
			\item \textit{Benutzername ist registriert, Passwort ist falsch} \\ $\implies$ Fehlermeldung wird angezeigt.
			\item \textit{Benutzername ist nicht registriert} \\ $\implies$ Fehlermeldung wird angezeigt.
		\end{enumerate}
	\item [Relevante Systemfunktionen:] \ref{funk:zugriff}
	\item [Relevante Use Cases:] \ref{uc:usermanagement}
\end{lhp}

\begin{lhp}{tf}{TF}{tests:abmelden}
	\item [Name:] Spieler abmelden.
	\item [Motivation:] Testet, ob die Abmeldung vom System korrekt funktioniert.
	\item [Sczenarien:] \hfill
		\begin{enumerate}
			\item \textit{User ist eingeloggt} \\ $\implies$ Spieler wird in die Lobby bewegt und ist nun ausgeloggt.
			\item \textit{User ist ausgeloggt} \\ $\implies$ Fehlermeldung wird angezeigt.
		\end{enumerate}
	\item [Relevante Systemfunktionen:] \ref{funk:zugriff}
	\item [Relevante Use Cases:] \ref{uc:usermanagement}
\end{lhp}

\begin{lhp}{tf}{TF}{tests:spielerstellen}
	\item [Name:] Ein Spiel erstellen.
	\item [Motivation:] Testet, ob ein Spiel erstellt werden kann.
	\item [Sczenarien:] \hfill
		\begin{enumerate}
			\item \textit{User ist eingeloggt} \\ $\implies$ Spieler wird in die Lobby des erstellten Spiels bewegt.
			\item \textit{User ist ausgeloggt} \\ $\implies$ Fehlermeldung wird angezeigt.
		\end{enumerate}
	\item [Relevante Systemfunktionen:] \ref{funk:spielraum}
	\item [Relevante Use Cases:] \ref{uc:spielstarten}
\end{lhp}

\begin{lhp}{tf}{TF}{tests:spielbeitreten}
	\item [Name:] Einem Spiel beitreten.
	\item [Motivation:] Testet, ob einem Spiel beigetreten werden kann.
	\item [Sczenarien:] \hfill
		\begin{enumerate}
			\item \textit{User ist eingeloggt} \\ $\implies$ Spieler wird in die Lobby des Spiels bewegt.
			\item \textit{User ist ausgeloggt} \\ $\implies$ Fehlermeldung wird angezeigt.
			\item \textit{Spiel existiert nicht} \\ $\implies$ Fehlermeldung wird angezeigt.
			\item \textit{Spiel läuft bereits} \\ $\implies$ Fehlermeldung wird angezeigt.
		\end{enumerate}
	\item [Relevante Systemfunktionen:] \ref{funk:spielraum}
	\item [Relevante Use Cases:] \ref{uc:spielstarten}
\end{lhp}

\begin{lhp}{tf}{TF}{tests:spielzüge}
	\item [Name:] Testen der Spielzüge.
	\item [Motivation:] Testet mögliche Spielzüge, die ein Spieler machen kann.
	\item [Sczenarien:] \hfill
		\begin{enumerate}
			\item \textit{Legaler Spielzug wird gespielt} \\ $\implies$ Der Spielzug wird durchgeführt.
			\item \textit{Illegaler Spielzug wird gespielt} \\ $\implies$ Fehlermeldung wird angezeigt und der Spielzug wird nicht durchgeführt.
		\end{enumerate}
	\item [Relevante Systemfunktionen:] \ref{funk:spielverw}
	\item [Relevante Use Cases:] \ref{uc:spielstarten}
\end{lhp}

\begin{lhp}{tf}{TF}{tests:chat}
	\item [Name:] Chatfunktion.
	\item [Motivation:] Testet die Chatfunktion.
	\item [Sczenarien:] \hfill
		\begin{enumerate}
			\item \textit{Absenden einer Nachricht} \\ $\implies$ Nachricht wird gesendet und ist von den anderen clients empfangen.
			\item \textit{Ausfall der Chatfunktion} \\ $\implies$ Fehlermeldung wird angezeigt.
		\end{enumerate}
	\item [Relevante Systemfunktionen:] \ref{funk:spielraum}
	\item [Relevante Use Cases:] \ref{uc:spielstarten}
\end{lhp}

\begin{lhp}{tf}{TF}{tests:einstellungen}
	\item [Name:] Einstellungen.
	\item [Motivation:] Testet die Einstellungen und ob diese übernommen werden.
	\item [Sczenarien:] \hfill
		\begin{enumerate}
			\item \textit{Änderung einer Einstellung und drücken von übernehmen} \\ $\implies$ Von der Einstellung gewünschter Effekt ist sichtbar und / oder verifizierbar.
			\item \textit{Änderung einer Einstellung und schließen des Einstellungsfensters} \\ $\implies$ Einstellungen sind nicht übernommen.
		\end{enumerate}
	\item [Relevante Systemfunktionen:] \ref{funk:einstellung}
	\item [Relevante Use Cases:] \ref{uc:einstellungen}
\end{lhp}

\end{description}
\section*{Aufgabe 1}
\subsection*{a)}
Löschungen:\\
Klinik: (4,'Dummy-Klinik')
\subsection*{b)}
Löschungen:\\
Klinik: (2,'Alpecinklinik Hamburg');
Chirurg: (2, 'Dr. Klenk', 2);
Patient: (2, 'John Doe', 2);
Behandlung: (2,2);
\subsection*{c)}
Löschungen:\\
Klinik: (3,'Klinik G-Funk');\\
Abänderungen:\\
Patient: (3, 'Jane Smith', NULL);
\subsection*{d)}
Löschungen:\\
Klinik: (1,'Backhaus Oetker');\\
Abänderungen:\\
Chirurg: (1, 'Dr. Oetker', NULL) (3, 'Dr. Dre', NULL);
Patient: (1, 'Max Mustermann', NULL);
\subsection*{e)}
Löschungen:\\
Klinik: (1, 'Backhaus Oetker');\\
Chirurg: (1, 'Dr. Oetker', 1) (3, 'Dr. Dre', 1);\\
Patient: (1, 'Max Mustermann', 1);
Behandlung: (1,1) (1,3);
\subsection*{f)}
Löschungen:\\
Klinik: (1, 'Backhaus Oetker');\\
Chirurg: (1, 'Dr. Oetker', 1) (3, 'Dr. Dre', 1);\\
Patient: (1, 'Max Mustermann', 1);
Behandlung: (1,1);
Abänderungen:\\
Behandlung: (1, NULL) - War vorher Eintrag (1,3)
\subsection*{g)}
Löschungen:\\
Klinik: (1, 'Backhaus Oetker');\\
Chirurg: (1, 'Dr. Oetker', 1) (3, 'Dr. Dre', 1);\\
Patient: (1, 'Max Mustermann', 1);
Behandlung: (1,1) (1,3);
\subsection*{h)}
Löschungen:\\
Klinik: (3, 'Klinik G-Funk');\\
Patient: (3, 'Jane Smith', 3);\\
Abänderungen:\\
Behandlung: (1, NULL) - War vorher Eintrag (1,3);
\subsection*{i)}
Wird nicht durchgeführt, da ref(D) DELETE RESTRICT ist und Patient Jane Smith gelöscht wird (wegen DELETE CASCADE). Es existiert jedoch der Eintrag (1,3) in Behandlung, weshalb es nicht funktioniert.\\
\subsection*{j)}
Es können reihenfolgenahängige Ergebnisse auftreten.\\
Fall 1: Klinik -> Chirurg -> Behandlung -> Patient\\
Löschungen:\\
Klinik: (1, 'Backhaus Oetker');\\
Chirurg: (1, 'Dr. Oetker', 1) (3, 'Dr. Dre', 1);\\
Behandlung: (1,1) (1,3);\\
Patient: (1, 'Max Mustermann', 1);\\\\
Fall 2: Klinik -> Patient -> Behandlung -> Chirurg\\
Schlägt fehl, da Max Mustermann gelöscht wird, jedoch in Behandlung noch ein Eintrag existiert, welcher Max Mustermann referenziert.\\

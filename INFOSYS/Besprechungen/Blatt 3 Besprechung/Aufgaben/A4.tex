\section*{Aufgabe 0 vom nächsten Blatt}
\subsection*{a)}
\subsubsection*{i)}
Domänenkalkül (keine Punktnotation, subnotation verwenden):
\begin{align*}
&\{p.pnr,p.vn,p.nn\ |\ \exists p.gehalt \exists p.abteilung | \in P(p.pnr,p.gehalt,p.vn,p.nn,p.abteilung,\_)\\
&\land \exists m.pnr M(m.pnr,\_,p.abteilung)\land p(m.pnr,p.gehalt,\_,\_,\_,\_)\}
\end{align*}
Tupelkalkül:
\begin{align*}
  &\{p.pnr,p.vn,p.nn\ |\ p\in P\land \exists m\in M(m.leitet = p.abteilung)\land\\ 
  &\exists mp\in P(p.pnr = m.pnr \land p.gehalt = mp.gehalt)\}
\end{align*}
\subsubsection*{ii)}
Domänenkalkül:
\begin{align*}
  \{a_{anr},a_{name}\ |\ A(a_{anr},a_{name},\_,\_)\land \exists m_1 (M(m_1,\_,a_{anr})\land \text{not }\exists m_2(M(m_2,\_,a_{anr})\land m_1 \not = m_2))  \}
\end{align*}
Tupelkalkül:
\begin{align*}
&\{a.anr,a.name\ |\ a\in A \land \exists m_1 \in M(a.anr = m.leitet) \land\\
&\text{not }\exists m_2 \in M(a.anr = m_2.leitet \land m_1.pnr \not = m_2.pnr))\}
\end{align*}
\subsection*{b)}
Ist nur möglich wenn wir wissen wieviele Abteilungen ein Manager leitet bzw. wieviele Abteilungen es gibt.
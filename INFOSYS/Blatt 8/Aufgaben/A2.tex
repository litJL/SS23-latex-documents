\pagebreak
\section*{Aufgabe 2}
\subsection*{a)}
\subsubsection*{i)}
Die Funktionale Abhängigkeit $ C \rightarrow B $ ist nicht erfüllt, da z.B. bei t1 und t2 \\C = c6 ist, jedoch B = b2 und B = b8.
\subsubsection*{ii)}
Die Funktionale Abhängigkeit $ A \rightarrow C $ ist nicht erfüllt, da z.B. bei t0 und t9 \\A = a1 ist, jedoch C = c3 und C = c6.
\subsubsection*{iii)}
Die Funktionale Abhängigkeit $ CB \rightarrow A $ ist nicht erfüllt, da z.B. bei t8 und t9 \\B = b3 und C = c6 ist, jedoch A = a4 und A = a1.
\subsubsection*{iv)}
Die Funktionale Abhängigkeit $ BCD \rightarrow A $ ist nicht erfüllt, da z.B. bei t8 und t9 \\B = b3, C = c6 und D = d4 ist, jedoch A = a4 und A = a1.
\subsection*{b)}
\subsubsection*{i)}
\{C\} ist kein Superschlüssel, da bei t1 und t2 C = c6 ist, jedoch B = b2 und B = b8.
\subsubsection*{ii)}
\{D\} ist kein Superschlüssel, da bei t0 und t3 D = d5 ist, jedoch C = c3 und C = c6.
\subsubsection*{iii)}
\{A,D\} ist kein Superschlüssel, da bei t6 und t7 A = a2 und D = d2 ist, jedoch B = b6 und B = b7.
\subsubsection*{iv)}
\{B,C\} ist kein Superschlüssel, da bei t4 und t6 B = b6 und C = c4 ist, jedoch D = d4 und D = d2.
\subsubsection*{v)}
\{A,B,C,D\} ist ein Superschlüssel, da alle Attribute in der Relation vorkommen und alle Attribute zusammen eindeutig sind. Es ist jedoch kein Kandidatenschlüssel, da \{A,B,D\} auch ein Superschlüssel ist.
\subsubsection*{vi)}
\{A,B,D\} ist ein Superschlüssel, da es keine Tupel gibt, bei denen A, B und D gleich sind. Zudem ist es ein Kandidatenschlüssel, denn es gilt: \\
\{A,D\} haben wir bereits als kein Superschlüssel identifiziert. \\
\{B,D\} ist kein Superschlüssel, da bei t8 und t9 B = b3 und D = d4 ist, jedoch A = a4 und A = a1. \\
\{A,B\} ist kein Superschlüssel, da bei t4 und t6 A = a2 und B = b6 ist, jedoch D = d4 und D = d2. \\
Und folglich sind natürlich \{A\}, \{B\} und \{D\} keine Superschlüssel.




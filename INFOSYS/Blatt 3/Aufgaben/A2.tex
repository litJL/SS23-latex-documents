\pagebreak
\section*{Aufgabe 2}
Angenommen wir haben folgende Relationen:
\begin{align*}
  &R(A_1,\dots,A_m,B_1,\dots,B_k), \\
  &S(B_1,\dots,B_k,C_1,\dots,C_n), \\
  &H(C_1,\dots,C_n,D_1,\dots,D_p)
\end{align*}
Wir verwenden die Definition des natürlichen Verbunds aus der Vorlesung:
\begin{align*}
  R \Join S(A_1,\dots,A_m,{\color{darkgreen} B_1,\dots,B_k},C_1,\dots,C_n)
\end{align*}
\subsection*{Assoziativität:}
\subsubsection*{Für $R \Join (S \Join H)$: }
\begin{align*}
  S\Join H(B_1,\dots,B_k,C_1,\dots,C_n,D_1,\dots,D_p)
\end{align*}
Da in $S\Join H$ immernoch $B_1,\dots,B_k$ enthalten ist, gilt laut Definition:
\begin{align*}
  R \Join (S \Join H)(A_1,\dots,A_m,B_1,\dots,B_k,C_1,\dots,C_n,D_1,\dots,D_p)
\end{align*}
\subsubsection*{Für $(R \Join S) \Join H$: }
\begin{align*}
  R\Join S(A_1,\dots,A_m,B_1,\dots,B_k,C_1,\dots,C_n)
\end{align*}
Da in $R\Join S$ immernoch $C_1,\dots,C_n$ enthalten ist, gilt laut Definition:
\begin{align*}
  (R \Join S) \Join H(A_1,\dots,A_m,B_1,\dots,B_k,C_1,\dots,C_n,D_1,\dots,D_p)
\end{align*}
Hiermit ist klar zu sehen, dass:
\begin{align*}
  R \Join (S \Join H) = (R \Join S) \Join H
\end{align*}
\subsection*{Kommutativität:}
\subsubsection*{Für $R\Join S$:}
\begin{align*}
  R\Join S(A_1,\dots,A_m,B_1,\dots,B_k,C_1,\dots,C_n)
\end{align*}
\subsubsection*{Für $S\Join R$:}
Da die Reihenfolge der Attribute keine Rolle spielt, gilt:
\begin{align*}
  S\Join R(A_1,\dots,A_m,B_1,\dots,B_k,C_1,\dots,C_n)
\end{align*}
Also muss gelten:
\begin{align*}
  R\Join S = S\Join R
\end{align*}

